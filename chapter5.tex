%%% Local Variables: ***


\chapter{Measuring the Quenching Timescales of Milky Way Satellites with \emph
  {Gaia} Proper Motions}





\section{Introduction}
\label{sec:intro} 


Large surveys of the nearby Universe show that galaxies at (and below) the mass
scale of the Magellanic clouds
($10^{6}~\msun \lesssim \mstar \lesssim 10^{9}~\msun$) are predominantly star
forming in the field \citep{haines08, geha12}.
%
This stands in stark contrast to the population of low-mass galaxies that
currently reside near a more massive host system, where the fraction of systems
that are no longer forming stars (i.e.~``quenched'') is significantly larger 
% dramatically larger
\citep{geha12, wheeler14, phillips15a}.
%
For dwarfs with $\mstar \lesssim 10^{8}~\msun$ in the Local Volume, this
field-satellite dichotomy is very apparent, with satellites of the Milky Way
(MW) and M31 being largely gas-poor, passive systems in contrast to the
gas-rich, star-forming field population \citep[e.g.][]{mateo98, grcevich09,
  spekkens14}.
%
This clear distinction between the field and satellite populations strongly
favors environmental processes as the dominant quenching mechanisms in this
low-mass regime \citep[$\mstar \lesssim 10^{8}~\msun$,][]{lin83, slater14,
  weisz15, wetzel15b, fham15, fham16, fham18}.
%
At the very lowest mass scales (i.e.~the regime of ultra-faint dwarfs), however,
there is evidence for a transition in the dominant quenching mechanism from one
associated with galaxy environment to one driven by reionization. 
%
The universally old stellar populations observed in the ultra-faint dwarfs
(UFDs) suggest that star formation on the smallest galactic scales is suppressed
at very early times and across all environments, marking a significant
transition in the dominant quenching mechanism at the lowest masses
\citep{brown14, wimberly19}.
%



Across all mass scales, some of the most powerful studies of satellite quenching
have utilized measurements of satellite and field quenched fractions to infer
the timescale upon which satellite quenching occurs following infall
\citep[e.g.][]{wetzel13, fham15, balogh16, fossati17}.
%
These studies point to a picture where quenching proceeds relatively slowly at
high satellite masses, consistent with quenching via starvation
\citep{fham15}. Below some host-dependent critical mass scale, however,
quenching is rapid, as stripping becomes increasingly efficient \citep{fham16}.
%
In defining this model of satellite quenching, simulations are commonly utilized
to constrain the distribution of infall times for an observed sample of
satellites. This statistical approach is required, as it is extremely difficult
to infer the infall time for a significant fraction of the satellite population
in even the most nearby groups and clusters.
%
Moreover, in systems more distant than $\sim1$~Mpc, it is difficult to measure a
precise star-formation history via spatially-resolved stellar photometry, even
with the aid of imaging from the {\it Hubble Space Telescope} ({\it HST}).
%
Within the LG, however, we are afforded the luxury of more detailed observations
of the nearby satellite and field populations. This is particularly true with
the release of {\it Gaia} Data Release 2 \citep[DR2,][]{gaia, gaiaDR2}, which
now enables an investigation of satellite quenching timescales (measured
relative to infall) on an object-by-object basis.
%
This offers a unique opportunity to test the results of large statistical
analyses and our current physical picture of satellite quenching. 


In this work, we aim to determine the quenching timescale and ultimately
constrain the potential mechanisms responsible for suppressing star formation in
individual MW satellite galaxies.
%
Utilizing the latest data products from {\it Gaia} DR2 \citep{gaiaDR2,
  gaiaDR2PM}, we infer the cosmic time when each dwarf galaxy around the MW
became a satellite (i.e.~the infall time) through comparison to cosmological
$N$-body simulations.
%
In addition, we infer the quenching times for the MW satellites based on their
published star-formation histories, as derived from {\it HST} imaging
\citep{weisz14a, weisz15, brown14}. Finally, through comparison of the quenching
times to the infall times, we will characterize the quenching timescales for
each object and constrain the potential mechanisms responsible for quenching
each MW satellite galaxy.
%
In \S\ref{sec:MWdwarfs}, we discuss our sample of local dwarfs and the
methodology by which we measure the infall and quenching time for each system.
%
Our primary results are presented in \S\ref{sec:results}, followed by a
discussion of how these results connect to physical models of satellite
quenching in \S\ref{sec:disc}. Finally, we summarize our results and
conclusions in \S\ref{sec:summary}.
%
Where necessary, we adopt a $\Lambda$CDM cosmology with the following
parameters: $\sigma_8 = 0.815$, $\Omega_{m} =
0.3121$, $\Omega_{\Lambda} = 0.6879$, $n_{s} = 0.9653$, and $h =
0.6751$ \citep{planck16}, consistent with the simulations used in this
work.


%%%%%%%%%%%%%%%%%%%%%%%%%%%%%%%%%%%%%%%%%%%%%%%%%

\begin{table*}
  \centering
  \begin{tabular}{lcccccccccc}
    %\begin{tabular}{l c c c c c} 
    \hline\hline
    $\rm Name$ & $\rm Distance$ & $V_{\rm 3D}$ & $V_{\rm r}$ & $\it e$
    & $t_{90}$ &  $t^{\rm upper}_{90}$ & $t^{\rm lower}_{90}$ &
                                                                $t_{\rm
                                                                infall}$
    &  $t^{\rm upper}_{\rm infall}$ &  $t^{\rm lower}_{\rm infall}$ \\

     & (1) & (2) & (3) & (4) & (5) & (6) & (7) & (8) & (9) & (10)  \\
    
  \hline
  \hline
     Sgr\_I &      19.0 &  316.31 &  140.82 &  0.42 &   3.40 &       5.23 &       3.08 &   10.71 &         11.83 &          9.54 \\
   Tuc\_III &      23.0 &  241.56 & -232.31 &  0.85 &     -- &         -- &         -- &   10.94 &         12.04 &          9.80 \\
     Hyd\_I &      26.0 &  413.93 &  -52.77 &  0.82 &     -- &         -- &         -- &    1.12 &          2.09 &          0.17 \\
    Dra\_II &      26.0 &  370.92 & -156.38 &  0.57 &     -- &         -- &         -- &    9.71 &         11.09 &          6.01 \\
     Seg\_1 &      28.0 &  233.12 &  116.48 &  0.37 &     -- &         -- &         -- &   10.78 &         11.84 &          9.73 \\
   Car\_III &      29.0 &  388.61 &   46.26 &  0.58 &     -- &         -- &         -- &    7.22 &          8.46 &          5.84 \\
     Seg\_2 &      32.0 &   70.84 &   50.75 &  0.87 &     -- &         -- &         -- &   10.76 &         11.86 &          9.54 \\
    Ret\_II &      32.0 &  231.60 & -102.07 &  0.32 &     -- &         -- &         -- &   10.99 &         12.08 &          9.81 \\
    Tri\_II &      36.0 &  335.57 & -255.14 &  0.70 &     -- &         -- &         -- &    9.85 &         11.07 &          6.11 \\
    Car\_II &      37.0 &  356.19 &  203.77 &  0.64 &     -- &         -- &         -- &    7.86 &         11.17 &          6.28 \\
  U\_Maj\_II &      38.0 &  258.39 &  -59.31 &  0.19 &     -- &         -- &         -- &   11.10 &         12.18 &          9.88 \\
    Boo\_II &      39.0 &  394.89 &  -47.85 &  0.68 &     -- &         -- &         -- &    1.02 &          1.99 &          0.06 \\
     Wil\_1 &      43.0 &  123.08 &   18.05 &  0.53 &     -- &         -- &         -- &   10.79 &         11.86 &          9.67 \\
  ComBer\_I &      45.0 &  285.72 &   29.06 &  0.24 &  13.05 &      14.30 &      11.80 &   10.43 &         11.62 &          9.01 \\
    Tuc\_II &      54.0 &  285.79 & -186.94 &  0.57 &     -- &         -- &         -- &    8.63 &         10.61 &          5.91 \\
     Boo\_I &      64.0 &  194.69 &   94.81 &  0.40 &  12.65 &      13.70 &      11.60 &   10.79 &         11.93 &          9.64 \\
     Dra\_I &      76.0 &  160.50 &  -88.41 &  0.53 &  10.15 &      11.68 &       7.66 &   10.82 &         12.01 &          9.38 \\
   U\_Min\_I &      78.0 &  154.43 &  -71.46 &  0.50 &   9.07 &      10.67 &       5.79 &   10.82 &         12.01 &          9.38 \\
     Hor\_I &      80.0 &  198.80 &  -34.09 &  0.16 &     -- &         -- &         -- &   10.22 &         11.45 &          8.83 \\
     Scu\_I &      86.0 &  197.16 &   74.70 &  0.33 &  10.61 &      11.90 &       7.08 &   10.68 &         11.98 &          6.24 \\
    Sext\_I &      89.0 &  243.13 &   79.23 &  0.30 &     -- &         -- &         -- &    9.11 &         10.92 &          6.28 \\
   U\_Maj\_I &     102.0 &  261.45 &   11.00 &  0.31 &  11.25 &      12.50 &      10.00 &    7.50 &          9.07 &          5.94 \\
    Aqu\_II &     105.0 &  294.26 &   43.31 &  0.50 &     -- &         -- &         -- &    1.43 &          2.40 &          0.48 \\
     Car\_I &     107.0 &  166.44 &    1.32 &  0.24 &   2.24 &       3.73 &       2.17 &    9.94 &         11.60 &          7.83 \\
     Gru\_I &     116.0 &  299.63 & -202.68 &  0.75 &     -- &         -- &         -- &    0.76 &          1.78 &          0.00 \\
    Cra\_II &     116.0 &  115.54 &  -83.51 &  0.73 &     -- &         -- &         -- &   10.59 &         11.74 &          8.25 \\
     Her\_I &     126.0 &  211.67 &  150.74 &  0.59 &  11.85 &      13.20 &      10.50 &    6.52 &         10.43 &          5.40 \\
    Hyd\_II &     131.0 &  257.21 &  129.03 &  0.55 &     -- &         -- &         -- &    1.57 &          2.55 &          0.60 \\
     Cra\_I &     145.0 &   89.34 &  -13.19 &  0.65 &     -- &         -- &         -- &   10.34 &         11.63 &          8.73 \\
     Frn\_I &     149.0 &  159.71 &  -41.34 &  0.28 &   2.24 &       2.51 &       2.04 &    7.37 &         10.75 &          5.93 \\
    Leo\_IV &     155.0 &  355.66 &   12.79 &  0.98 &  12.15 &      13.60 &      10.70 &   10.26 &         11.45 &          8.81 \\
 CanVen\_II &     161.0 &  217.80 &  -92.93 &  0.40 &  12.70 &      14.30 &      11.10 &    1.18 &          2.13 &          0.21 \\
    Pis\_II &     182.0 &  390.20 &  -79.19 &  0.99 &     -- &         -- &         -- &    6.91 &         10.85 &          5.61 \\
     Leo\_V &     197.0 &  398.37 &   42.05 &  0.99 &     -- &         -- &         -- &    7.08 &         10.13 &          5.35 \\
  CanVen\_I &     218.0 &  142.78 &   83.72 &  0.55 &   8.32 &       9.46 &       6.31 &    7.25 &          8.66 &          5.82 \\
    Leo\_II &     236.0 &  118.81 &   16.05 &  0.43 &   6.41 &       7.16 &       5.81 &    7.48 &          9.07 &          5.83 \\
     Leo\_I &     257.0 &  183.48 &  168.25 &  0.85 &   1.68 &       1.89 &       1.62 &    2.45 &          3.45 &          1.44 \\
%    Eri\_II &     382.0 &  792.08 &  -61.16 &  0.99 &     -- &         -- &         -- &    0.15 &          1.11 &          0.00 \\
%     Phe\_I &     415.0 &  134.56 & -116.11 &  0.79 &     -- &         -- &         -- &    6.21 &          7.48 &          5.16 \\

\hline

\label{table:MWdwarfs} 
\end{tabular} 
\caption{
  The properties of satellite galaxies of the Milky Way, as selected from 
  \citet{fritz18} and listed in order of increasing Galactocentric distance.
  (1) The distance from the center of the MW in kpc
  \citep{fritz18}.
  (2)  The total velocity, in ${\rm km}~{\rm s}^{-1}$ in the
  Galactocentric frame-of-reference \citep{fritz18}.
  (3)  The radial velocity, in ${\rm km}~{\rm s}^{-1}$ in the
  Galactocentric frame-of-reference \citep{fritz18}.
  (4)  The eccentricity, $e$, based on the ``heavy'' MW mass used
  in \citet{fritz18}.
  (5) The quenching time, $t_{90}$, in Gyr, inferred from the SFHs by adopting the
  lookback time at which the dwarf galaxy formed $90\%$ of its current
  stellar mass \citep{weisz14a, brown14}.
  (6,7) The $1\sigma$ bounds on the quenching time, in Gyr, inferred
  from the SFHs by adopting the lookback-time at which the $1\sigma$
  bounds on the SFH crossed the $90\%$ threshold.
  (8) The infall time, in Gyr, from matching the observed binding energy to the
  subhalo binding energy distribution in the simulations.
  (9,10) The uncertainty in the infall time, in Gyr, as determined by the full-width half-maximum of the
  infall time distribution.
  This table will be available at https://sfillingham.github.io after the paper is
  accepted for publication. }

\end{table*} 

%%%%%%%%%%%%%%%%%%%%%%%%%%%%%%%%%%%%%%%%%%%%%%%%%




