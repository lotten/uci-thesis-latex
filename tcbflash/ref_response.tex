

\documentclass[11pt, preprint]{aastex}
\renewcommand{\baselinestretch}{0.95}

\setlength{\oddsidemargin}{-0.2 in}
\setlength{\evensidemargin}{-0.2 in}
\setlength{\topmargin}{-0.75 in}
\setlength{\textheight}{9.0 in}
\setlength{\textwidth}{6.5 in}
\setlength{\marginparwidth}{0 in}

\usepackage{times}
\usepackage{amsmath}
\usepackage{hyperref}
\usepackage{natbib}
\usepackage{fancyhdr}
\pagenumbering{arabic} %roman/Roman/arabic


\begin{document}
\pagestyle{empty}

\hspace*{0.5in} \vspace*{0.5in}

\noindent Dear MNRAS Editorial Staff, 

\noindent We again thank the referee for carefully rereading the paper and for recommending useful changes that will improve the clarity of our work. We agree that our original analysis of ram-pressure stripping was incomplete. The primary goal of this paper is to illustrate the short quenching timescales at low masses and secondarily show (very roughly) that this could be due to ram-pressure stripping. As the referee suggests, to do the latter requires a more quantative analysis that takes into account the range of radial gas distributions within field dwarfs at these masses and a more involved estimation of the host halo density profiles. For this reason, we have removed the back-of-the-envelope calculation associated with the efficiency of ram-pressure stripping from the manuscript in favor of a more thorough analysis in a follow-up paper. The discussion of quenching mechanisms below $\sim10^{8}~{\rm M}_{\odot}$ has been revised accordingly, with a much more speculative tone and presenting ram-pressure stripping briefly as one of several potential processes at play.  

\noindent We thank the referee for their detailed review. It has pushed us to pursue a more complete and meaningful analysis of ram-pressure stripping in the Local Group. As suggested by the referee's comments, it is a topic and analysis requiring a more thorough treatment and thus a separate paper. In the interest of clarity, we have answered the referee's questions below (with our responses indented and interspersed with the original report); we will be sure to include more detailed information on these points in our follow-up paper.

\noindent Best, \\
\noindent Sean Fillingham and Mike Cooper \\

\vspace*{0.3in}

\hrule

\noindent Reviewer's Comments: 

\noindent Comments to the Author

\noindent I thank the authors for their response to my previous report. There is only one response which I believe requires further elaboration.

\noindent Title: ``Taking Care of Business in a Flash: Constraining the Timescale for Low-Mass Satellite Quenching in ELVIS'' Author: Fillingham, Cooper, Wheeler, Garrison-Kimmel, Boylan-Kolchin, Bullock

\noindent A key point of their paper is the claim that satellites with stellar masses greater than $>10^8$ Msun are quenched through strangulation processes while below that mass scale further ram pressure stripping of the cold gas in the galaxy occurs. Previously, the authors had shown that ram pressure stripping was effective below $10^8$ Msun, but not that it was not efficient above that mass. The authors response was: ``This is a very good point. We have added text to Section 5 to address the effectiveness of ram-pressure stripping at higher masses. Using estimates for the HI surface density in field systems at M? $>10^8$ Msun, we show that ram pressure will only remove cold gas from the outskirts of more massive systems, having minimal impact on the effective depletion timescale (and thus quenching timescale).''



\noindent It appears that the statement the authors have added to section 5 is the following: ``Observations of HI in field dwarfs with M*$>10^8$ Msun generally show extended atomic gas distributions with typical surface densities of $>10^(19.5)$ atoms $cm^-2$ out to radial distances of $\sim5$ kpc (Hunter et al.~2012). Given the expected dark matter halo masses for such systems (sigma $> 50$km/s), ram-pressure will only be effective at rather large radii ($>5 kpc$), where a small percentage of their gas reservoir resides."

\noindent As the authors 'stripping' equation (Eq 4) is self-similar, the difference in the effectiveness in stripping and strangulation must arise because of the change in the relative distribution of HI in dwarfs at this mass scale. As this is the main point of the paper, it is necessary to be more careful and explicit than the authors currently are. My main questions are:

\noindent 1. The Hunter et al.~(2012) makes no mention of stellar masses, so which M/L ratios or stellar mass estimates did they use to define those galaxies within the sample that are M*$>10^8$.

\indent \parbox{5.5in}{We are employing a ${\rm M}/L_{V}$ of 1, utilizing the $V$-band luminosities published by Hunter et al.~(2012).}

\noindent 2. How many of these galaxies ``generally show extended atomic gas distributions with typical surface densities...''?

\indent \parbox{5.5in}{This is an excellent question. We intend to address this in our follow-up paper, where we will anlayze the HI profiles from Little THINGS in more detail. Our initial work suggests that there is a broad range of gas density profiles, which (as the referee suggests) complicates our back-of-the-envelope calculations for stripping.}

\noindent 3. What is the distribution of percentage of gas reservoir outside of 5 kpc and how does that compare to the distribution of the percentage of gas reservoir outside of 1 kpc in $10^7$ Msun galaxies.

\indent \parbox{5.5in}{Our ongoing work is geared towards addressing this exact question. By constraining the fraction of total gas stripped, we will be able to estimate the expected reduction in the quenching timescale for starvation.}

\noindent 4. Is it fair to assume the LITTLE THINGS sample is fully representative of galaxies at M*$>10^8$? 

\indent \parbox{5.5in}{We are currently working to compile HI profiles for massive dwarfs from other datasets, so as to broaden the sample in the high-mass regime.}






\vspace*{0.2in}


\end{document}