%%% Local Variables: ***
\newcommand{\vcirc}{v_{\rm{circ}}}
\newcommand{\vmax}{V_{\rm{max}}}
\newcommand{\rmax}{R_{\rm{max}}}
\newcommand{\mhalf}{M_{1/2}}
\newcommand{\rhalf}{R_{1/2}}
\newcommand{\mmax}{M_{\rm max}}
\newcommand{\msub}{M_{\rm{sub}}}
\newcommand{\mvir}{M_{\rm{vir}}}
\newcommand{\mhalo}{{M}_{\rm{halo}}}
\newcommand{\mpeak}{{M}_{\rm{peak}}}
\newcommand{\rvir}{R_{\rm{vir}}}
\newcommand{\vvir}{V_{\rm{vir}}}
\newcommand{\dd}{{\rm d}}
\newcommand{\mstar}{{\rm M}_{\star}}
\newcommand{\lstar}{L^{*}}
\newcommand{\msun}{{\rm M}_{\odot}}
\newcommand{\lsun}{L_{\odot}}
\newcommand{\mpc}{{\rm Mpc}}
\newcommand{\kpc}{{\rm kpc}}
\newcommand{\kms}{{\rm km \, s}^{-1}}
\newcommand{\millen}{MS-I}
\newcommand{\msii}{MS-II}
\newcommand{\lcdm}{$\Lambda$CDM}
\newcommand{\diso}{d_{\rm iso}}
\newcommand{\niso}{n_{\rm iso}}
\newcommand{\lt}{<}
\newcommand{\gt}{>}

\chapter{Introduction}


The bimodal nature of the galaxy population has been known for many
decades, with galaxies generally residing in one of two groups:
actively star forming (blue cloud) or no longer star forming (red
sequence).  
%
One of the principal results from large surveys in the past $\sim 15$
years is that the population of galaxies with suppressed (or
``quenched'') star formation has grown by roughly a factor of $2$
since $z \sim 2$. 
Resulting in a buildup of the red sequence at late cosmic times, such
that quenched galaxies dominate the $z \sim 0$ stellar mass budget
\citep{faber07}.  
%
Attempts to understand this transition from star forming to quenched
in both semi-analytic models (SAMs) and hydrodynamic simulations have
been somewhat successful, specifically in the most massive galaxies when
feedback from AGN are included.
%
%Despite these encouraging results in the most massive galaxies,
In stark contrast, the models clearly begin to break down in low-mass
galaxies, substantially overproducing the population of low-mass
quenched galaxies \citep{kimm09}. 



Closer inspection of the low-mass galaxy population shows that
centrals -- the most massive galaxy in their dark matter halo -- are
predominantly star forming and well matched by the theoretical models.
%
While a higher fraction of satellites -- smaller galaxies orbiting
within the virial radius of a more massive host -- are quenched with
the models grossly overpredicting their frequency \citep{geha12,
  hirschmann14}. 
%
This observed central-satellite dichotomy in the 
%star forming properties of the 
low-mass galaxy population strongly favors quenching
mechanisms associated with the host environment, such as ram-pressure
stripping \citep[RPS,][]{gunn72} and/or starvation \citep{larson80}.
%
Leading to the conclusion that the struggle to model the low-mass
galaxy population hinges on inept environmental quenching,
more specifically it hinges on our inability to account for the
\emph{efficiency} of environmental quenching mechanisms.
%



In the Local Group (LG), the central-satellite dichotomy is even more
pronounced with the field dwarf galaxies being almost exclusively gas
rich and star forming, while the satellite galaxies are predominantly
gas poor and quenched \citep{spekkens14,wheeler14}. 
%
The combination of very detailed observations and cosmological
zoom-in simulations makes the local Universe an ideal laboratory in
which to test and further constrain the environmental quenching
mechanisms operating on the low-mass satellite galaxy population.
%
%Much of my work thus far has focused on the dwarf galaxy population in
%the LG in an effort to understand the suppression of star formation on
%the smallest scales.
%
As I describe in more detail below, my work has focused on the dwarf
galaxy population in the LG and has led to a complete picture of
environmental quenching in group environments
\citep{fham15,fham16,fham18}.